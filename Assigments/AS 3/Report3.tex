% Options for packages loaded elsewhere
\PassOptionsToPackage{unicode}{hyperref}
\PassOptionsToPackage{hyphens}{url}
%
\documentclass[
  12pt,
]{article}
\usepackage{amsmath,amssymb}
\usepackage{lmodern}
\usepackage{setspace}
\usepackage{iftex}
\ifPDFTeX
  \usepackage[T1]{fontenc}
  \usepackage[utf8]{inputenc}
  \usepackage{textcomp} % provide euro and other symbols
\else % if luatex or xetex
  \usepackage{unicode-math}
  \defaultfontfeatures{Scale=MatchLowercase}
  \defaultfontfeatures[\rmfamily]{Ligatures=TeX,Scale=1}
\fi
% Use upquote if available, for straight quotes in verbatim environments
\IfFileExists{upquote.sty}{\usepackage{upquote}}{}
\IfFileExists{microtype.sty}{% use microtype if available
  \usepackage[]{microtype}
  \UseMicrotypeSet[protrusion]{basicmath} % disable protrusion for tt fonts
}{}
\makeatletter
\@ifundefined{KOMAClassName}{% if non-KOMA class
  \IfFileExists{parskip.sty}{%
    \usepackage{parskip}
  }{% else
    \setlength{\parindent}{0pt}
    \setlength{\parskip}{6pt plus 2pt minus 1pt}}
}{% if KOMA class
  \KOMAoptions{parskip=half}}
\makeatother
\usepackage{xcolor}
\usepackage[margin=1in]{geometry}
\usepackage{graphicx}
\makeatletter
\def\maxwidth{\ifdim\Gin@nat@width>\linewidth\linewidth\else\Gin@nat@width\fi}
\def\maxheight{\ifdim\Gin@nat@height>\textheight\textheight\else\Gin@nat@height\fi}
\makeatother
% Scale images if necessary, so that they will not overflow the page
% margins by default, and it is still possible to overwrite the defaults
% using explicit options in \includegraphics[width, height, ...]{}
\setkeys{Gin}{width=\maxwidth,height=\maxheight,keepaspectratio}
% Set default figure placement to htbp
\makeatletter
\def\fps@figure{htbp}
\makeatother
\setlength{\emergencystretch}{3em} % prevent overfull lines
\providecommand{\tightlist}{%
  \setlength{\itemsep}{0pt}\setlength{\parskip}{0pt}}
\setcounter{secnumdepth}{-\maxdimen} % remove section numbering
\usepackage{amssymb}
\usepackage{amsmath}
\usepackage{subfig}
\usepackage{booktabs}
\ifLuaTeX
  \usepackage{selnolig}  % disable illegal ligatures
\fi
\IfFileExists{bookmark.sty}{\usepackage{bookmark}}{\usepackage{hyperref}}
\IfFileExists{xurl.sty}{\usepackage{xurl}}{} % add URL line breaks if available
\urlstyle{same} % disable monospaced font for URLs
\hypersetup{
  pdftitle={Econ771 - Empirical Exercise 3},
  pdfauthor={Nixon Torres Candiales},
  hidelinks,
  pdfcreator={LaTeX via pandoc}}

\title{Econ771 - Empirical Exercise 3}
\author{Nixon Torres Candiales}
\date{November 4,2022}

\begin{document}
\maketitle

\setstretch{1.25}
\hypertarget{overview}{%
\section{Overview}\label{overview}}

In this assignment, we're going to work through some applied issues
related to regression discontinuity designs. We'll cover the basics of
strict and fuzzy RD, and we'll work through standard specification
tests. We'll also introduce some more technical aspects of bin and
bandwidth selection.

Please ``submit'' your answers as a GitHub repository link on Canvas. In
this repo, please include a final document with your main answers and
analyses in a PDF. Be sure to include in your repository all of your
supporting code files. Practice writing good code and showing me only
what I would need to recreate your results.

\hypertarget{resources-and-data}{%
\section{Resources and data}\label{resources-and-data}}

The data for this assignment comes from the AEJ: Policy website, where
Keith Ericson's complete dataset is available. The data are available
\href{https://www.aeaweb.org/articles?id=10.1257/pol.6.1.38}{here}. I
will also upload the replication files to our class OneDrive folder.

\hypertarget{questions}{%
\section{Questions}\label{questions}}

In your GitHub repository, please be sure to clearly address/answer the
following questions.

\begin{enumerate}
\def\labelenumi{\arabic{enumi}.}
\tightlist
\item
  Recreate the table of descriptive statistics (Table 1) from
  @ericson2014.
\end{enumerate}

\begin{table}

\caption{Descriptive Statistics of Medicare Part D Plans}
\centering
\begin{tabular}[t]{lccccc}
\toprule
\multicolumn{1}{c}{ } & \multicolumn{5}{c}{Cohort (Year of plan introduction)} \\
\cmidrule(l{3pt}r{3pt}){2-6}
  & 2006 & 2007 & 2008 & 2009 & 2010\\
\midrule
Mean monthly premium & \$37 & \$40 & \$36 & \$30 & \$33\\
 & (13) & (17) & (20) & (5) & (9)\\
Mean deductible & \$92 & \$114 & \$146 & \$253 & \$118\\
 & (116) & (128) & (125) & (102) & (139)\\
Fraction enhanced benefit & 0.43 & 0.43 & 0.58 & 0.03 & 0.69\\
\addlinespace[0.3em]
\multicolumn{6}{l}{Fraction of plans offered by firms already offering a plan ... }\\
\hspace{1em}... in the United States & 0 & 0.76 & 0.98 & 1 & 0.97\\
\hspace{1em}... in the same state & 0 & 0.53 & 0.91 & 0.68 & 0.86\\
Number of unique firms & 51 & 38 & 16 & 5 & 6\\
Number of plans & 1429 & 658 & 202 & 68 & 107\\
\bottomrule
\end{tabular}
\end{table}
\pagebreak 

\begin{enumerate}
\def\labelenumi{\arabic{enumi}.}
\setcounter{enumi}{1}
\item
  Recreate Figure 3 from @ericson2014.

  \includegraphics{Output/fig/Q2.png}
\end{enumerate}

\newpage

\begin{enumerate}
\def\labelenumi{\arabic{enumi}.}
\setcounter{enumi}{2}
\item
  @calonico2015 discuss the appropriate partition size for binned
  scatterplots such as that in Figure 3 of Ericson (2014). More
  formally, denote by
  \(\mathcal{P}_{-,n} = \{ P_{-,j} : j=1, 2, ... J_{-, n} \}\) and
  \(\mathcal{P}_{+,n} = \{ P_{+,j} : j=1, 2, ... J_{+, n} \}\) the
  partitions of the support of the running variable \(x_{i}\) on the
  left and right (respectively) of the cutoff, \(\bar{x}\). \(P_{-, j}\)
  and \(P_{+, n}\) denote the actual supports for each \(j\) partition
  of size \(J_{-,n}\) and \(J_{+,n}\), such that
  \([x_{l}, \bar{x}) = \bigcup_{j=1}^{J_{-,n}} P_{-, j}\) and
  \((\bar{x}, x_{u}] = \bigcup_{j=1}^{J_{+,n}} P_{+, j}\). Individual
  bins are denoted by \(p_{-,j}\) and \(p_{+,j}\). With this notation in
  hand, we can write the partitions \(J_{-,n}\) and \(J_{+,n}\) with
  equally-spaced bins as
  \[p_{-,j}=x_{l} + j \times \frac{\bar{x} - x_{l}}{J_{-,n}},\] and
  \[p_{+,j} = \bar{x} + j \times \frac{x_{u} - \bar{x}}{J_{+,n}}.\]
  Recreate Figure 3 from Ericson (2014) using \(J_{-,n}=J_{+,n}=10\) and
  \(J_{-,n}=J_{+,n}=30\). Discuss your results and compare them to your
  figure in Part 2.

  \includegraphics{Output/fig/Q3A.png}
  \includegraphics{Output/fig/Q3A2.png}

  \includegraphics{Output/fig/Q3B.png}

  \includegraphics{Output/fig/Q3B2.png}
\item
  With the notation above, @calonico2015 derive the optimal number of
  partitions for an evenly-spaced (ES) RD plot. They show that
  \[J_{ES,-,n} = \left\lceil \frac{V_{-}}{\mathcal{V}_{ES,-}} \frac{n}{\text{log}(n)^{2}} \right\rceil\]
  and
  \[J_{ES,+,n} = \left\lceil \frac{V_{+}}{\mathcal{V}_{ES,+}} \frac{n}{\text{log}(n)^{2}} \right\rceil,\]
  where \(V_{-}\) and \(V_{+}\) denote the sample variance of the
  subsamples to the left and right of the cutoff and
  \(\mathcal{V}_{ES,.}\) is an integrated variance term derived in the
  paper. Use the \texttt{rdrobust} package in \texttt{R} (or
  \texttt{Stata} or \texttt{Python}) to find the optimal number of bins
  with an evenly-spaced binning strategy. Report this bin count and
  recreate your binned scatterplots from parts 2 and 3 based on the
  optimal bin number.
\end{enumerate}

\includegraphics{Output/fig/Q4B.png}
\includegraphics{Output/fig/Q4A.png} \pagebreak

\begin{enumerate}
\def\labelenumi{\arabic{enumi}.}
\setcounter{enumi}{4}
\tightlist
\item
  Although in the online appendix we see evidence of a small
  discontinuity at the cutoff, after correcting by the optimal bin
  selection we reject the hypothesis of manipulation in the running
  variable.
\end{enumerate}

\includegraphics{Output/fig/Q5.png} \begin{table}

\caption{Table 5}
\centering
\begin{tabular}[t]{lc}
\toprule
  & Model 1\\
\midrule
time\_to\_treat = -7 × treated & \num{5.137}***\\
 & (\num{1.412})\\
time\_to\_treat = -6 × treated & \num{4.920}**\\
 & (\num{1.502})\\
time\_to\_treat = -5 × treated & \num{2.254}\\
 & (\num{1.551})\\
time\_to\_treat = -4 × treated & \num{-1.602}\\
 & (\num{1.308})\\
time\_to\_treat = -3 × treated & \num{-2.416}*\\
 & (\num{1.027})\\
time\_to\_treat = -2 × treated & \num{-0.141}\\
 & (\num{0.578})\\
time\_to\_treat = 0 × treated & \num{8.151}***\\
 & (\num{1.599})\\
time\_to\_treat = 1 × treated & \num{-16.830}***\\
 & (\num{1.009})\\
time\_to\_treat = 2 × treated & \num{-22.370}***\\
 & (\num{1.256})\\
time\_to\_treat = 3 × treated & \num{-27.793}***\\
 & (\num{1.537})\\
time\_to\_treat = 4 × treated & \num{-30.231}***\\
 & (\num{1.738})\\
time\_to\_treat = 5 × treated & \num{-33.524}***\\
 & (\num{2.060})\\
\midrule
Num.Obs. & \num{79557}\\
AIC & \num{805085.8}\\
BIC & \num{805206.5}\\
RMSE & \num{38.12}\\
Std.Errors & by: pn\\
FE: pn & X\\
FE: year & X\\
\bottomrule
\multicolumn{2}{l}{\rule{0pt}{1em}+ p $<$ 0.1, * p $<$ 0.05, ** p $<$ 0.01, *** p $<$ 0.001}\\
\end{tabular}
\end{table}


\newpage

\begin{enumerate}
\def\labelenumi{\arabic{enumi}.}
\setcounter{enumi}{5}
\tightlist
\item
  Recreate Table 3 of @ericson2014 using the same bandwidth of
  \$4.00.\newline
\end{enumerate}

We recreate the table as follows, note the standard errors differs at
the third decimal. This discrepancy might be due to a different default
parameter in R when calculating the standard errors.

\begin{center}
\begin{table}
\caption{Effect of LIS Benchmark Status in 2006 on Plan Enrollment}
\centering
\begin{tabular}[t]{lccccc}
\toprule
 $\ln s_t$ & 2006 & 2007 & 2008 & 2009 & 2010\\
\midrule
\multicolumn{6}{l}{\textit{Panel A. Local linear, bandwidth \$4}}\\
Below benchmark, 2006 & 2.224*** & 1.332*** & 0.902** & 0.803* & 0.677\\
 & (0.283) & (0.267) & (0.248) & (0.362) & (0.481)\\
\addlinespace[0.3em]
\multicolumn{6}{l}{Premium—subsidy, 2006}\\
\hspace{1em}Below benchmark & -0.014 & -0.077 & -0.073 & -0.170 & -0.215*\\
\hspace{1em} & (0.032) & (0.088) & (0.116) & (0.105) & (0.088)\\
\hspace{1em}Above benchmark & -0.142+ & -0.033 & 0.049 & 0.074 & 0.049\\
\hspace{1em} & (0.078) & (0.110) & (0.163) & (0.170) & (0.202)\\
Num.Obs. & 306 & 299 & 298 & 246 & 212\\
R2 & 0.576 & 0.325 & 0.131 & 0.141 & 0.124\\
\\
\multicolumn{6}{l}{\textit{Panel B. Polynomial with controls, bandwidth \$4}}\\
Below benchmark, 2006 & 2.464*** & 1.364*** & 0.872** & 0.351 & -0.277\\
 & (0.219) & (0.317) & (0.243) & (0.321) & (0.298)\\
Premium--subsidy, 2006 & Quadratic & Quadratic & Quadratic & Quadratic & Quadratic\\
Num.Obs. & 306 & 299 & 298 & 246 & 212\\
R2 & 0.794 & 0.576 & 0.472 & 0.535 & 0.685\\
\bottomrule
\bottomrule
\multicolumn{6}{l}{\rule{0pt}{1em}+ p $<$ 0.1, * p $<$ 0.05, ** p $<$ 0.01, *** p $<$ 0.001}\\
\end{tabular}
\end{table}
\end{center}

\newpage

\begin{enumerate}
\def\labelenumi{\arabic{enumi}.}
\setcounter{enumi}{6}
\item
  @calonico2020 show that pre-existing optimal bandwidth calculations
  (such as those used in @ericson2014) are invalid for appropriate
  inference. They propose an alternative method to derive minimal
  coverage error (CE)-optimal bandwidths. Re-estimate your RD results
  using the CE-optimal bandwidth (\texttt{rdrobust} will do this for
  you) and compare the bandwidth and RD estimates to that in Table 3 of
  @ericson2014.

  \begin{table}
  \caption{Rdrobust estimation with optimal bandwith}
  \centering
  \begin{tabular}[t]{lccccc}
  \toprule
  $\ln s_t$ & 2006 & 2007 & 2008 & 2009 & 2010\\
  \midrule
  \addlinespace[0.3em]
  \multicolumn{6}{l}{\textit{Panel A. Local linear}}\\
  \hspace{1em}Conventional estimate & -2.29 & 0.70 & 0.25 & -1.23 & -1.07\\
  \hspace{1em} & (0.55) & (0.69) & (0.48) & (0.59) & (0.88)\\
  \hspace{1em}Observations & 306 & 245 & 200 & 143 & \vphantom{3} 128\\
  \hspace{1em}H & 0.75 & 1.92 & 2.38 & 2.12 & 1.95\\
  \hspace{1em}Bin & 1.59 & 4.88 & 7.00 & 4.89 & 4.85\\
  \hspace{1em}Kernel & Uniform & Uniform & Uniform & Uniform & \vphantom{1} Uniform\\
  \\
  \hspace{1em}Conventional estimate & -2.51 & 0.96 & 0.48 & -1.20 & -1.17\\
  \hspace{1em} & (0.52) & (0.72) & (0.49) & (0.65) & (0.63)\\
  \hspace{1em}Observations & 306 & 245 & 200 & 143 & \vphantom{2} 128\\
  \hspace{1em}H & 0.91 & 2.24 & 2.29 & 2.79 & 4.41\\
  \hspace{1em}Bin & 1.72 & 4.49 & 6.24 & 5.35 & 9.67\\
  \hspace{1em}Kernel & Triangular & Triangular & Triangular & Triangular & \vphantom{1} Triangular\\
  \addlinespace[0.3em]
  \multicolumn{6}{l}{\textit{Panel B. Quadratic Polinomial}}\\
  \hspace{1em}Conventional estimate & -2.58 & 0.77 & 1.13 & -0.67 & -0.84\\
  \hspace{1em} & (0.62) & (0.89) & (0.67) & (0.94) & (1.04)\\
  \hspace{1em}Observations & 306 & 245 & 200 & 143 & \vphantom{1} 128\\
  \hspace{1em}H & 1.02 & 2.61 & 2.79 & 2.52 & 3.20\\
  \hspace{1em}Bin & 2.07 & 5.36 & 6.96 & 5.27 & 6.13\\
  \hspace{1em}Kernel & Uniform & Uniform & Uniform & Uniform & Uniform\\
  \\
  \hspace{1em}Conventional estimate & -2.89 & 0.93 & 0.92 & -0.64 & -1.06\\
  \hspace{1em} & (0.62) & (0.99) & (0.75) & (1.19) & (1.01)\\
  \hspace{1em}Observations & 306 & 245 & 200 & 143 & 128\\
  \hspace{1em}H & 1.04 & 2.62 & 2.77 & 2.35 & 4.02\\
  \hspace{1em}Bin & 1.99 & 5.10 & 6.02 & 4.33 & 6.64\\
  \hspace{1em}Kernel & Triangular & Triangular & Triangular & Triangular & Triangular\\
  \bottomrule
  \bottomrule
  \multicolumn{6}{l}{\textit{Note:} Robust standard erros in parenthesis}\\
  \end{tabular}
  \end{table}
\end{enumerate}

Although the coefficients seems to flip signs, this is due to the way
the package reports the results. We see how the bin choice increases
when the number of observations decreases as well as the bandwidth. The
choice of the kernel also affects the optimal bin selection as shown in
the table.

\newpage

\begin{enumerate}
\def\labelenumi{\arabic{enumi}.}
\setcounter{enumi}{7}
\tightlist
\item
  Now let's extend the analysis in Section V of @ericson2014 using IV.
  Use the presence of Part D low-income subsidy as an IV for market
  share to examine the effect of market share in 2006 on future premium
  changes.
\end{enumerate}

\begin{table}
\caption{ Effect of LIS Benchmark Status in 2006 on Premiums in Later Year}
\centering
\begin{tabular}[t]{lcccc}
\toprule
Premium - Subsidy & 2007 & 2008 & 2009 & 2010\\
\midrule
Conventional Coef &-0.935 & 1.267 & 2.276 & 0.629\\
Confidence Interval & [-2.764,1.225] & [-1.927,5.173] & [-0.537,6.863] & [-4.478,8.157]\\
Observations & 569 & 512 & 378 & 322\\
\bottomrule
\multicolumn{5}{l}{\textit{Note:} Replicaton table A.7 Online Appendix. Fuzzy Regression Discontinuity}\\
\end{tabular}
\end{table}

\begin{table}
\caption{Effect of Market Share in 2006 on Future Premium Changes}
\centering
\begin{tabular}[t]{lc}
\toprule
  & ln(monthly premium)\\
\midrule
$\hat{lnS}$ & -0.165***\\
 & (0.024)\\
\midrule
Num.Obs. & 4123\\
State FE & X \\
Year FE & X\\
\bottomrule
\end{tabular}
\end{table}

Note table 5 presents the estimates for the Fuzzy Regression
discontinuity emulating table A7 in the online appendix. Under this
setting we find no evidence whether falling above or below the benchmark
in 2006 had any effect on average premiums in the subsequent years, as
mentioned in Ericson (2014).

\newpage

\begin{enumerate}
\def\labelenumi{\arabic{enumi}.}
\setcounter{enumi}{8}
\tightlist
\item
  Discuss your findings and compare results from different binwidths and
  bandwidths. Compare your results in part 8 to the invest-then-harvest
  estimates from Table 4 in @ericson2014.
\end{enumerate}

We found similar results than those in table 4 in Ericson (2014). The
invest-then-harvest hypothesis seems to explain the price behavior in
older plans (on their 5th year). Also, we found that for plans in the
second year they seem to decrease in price which can be seen as and
``investing'' or increasing the market share to in later periods
increase prices. Finally, with regards to market share, we found a
negative effect, which might be in line with the hypotesis that new
plans are priced lower.

\begin{enumerate}
\def\labelenumi{\arabic{enumi}.}
\setcounter{enumi}{9}
\tightlist
\item
  Reflect on this assignment. What did you find most challenging? What
  did you find most surprising?
\end{enumerate}

While reading the paper's online appendix I found there was a small
discontinuity around the cutoff. However, after testing for manipulation
in the forcing variable using the optimal binwidht implemented in
rdrobust I found there was no evidence of bunching. This result was nos
expected and I need to run further checks and confirm I use the correct
sintax on my code to be on the safe there are no code mistakes tat biase
my estimates. Overall, the intuition of the RD designs is straight
forward.

I still find working with the tables in Rmarkdown challenging but
definitely I have seen improvements on my workflow. On the specifics of
this assignment, the Fuzzy RD was challenging to implement. Having acces
to the paper code and online appendix made the difference in the
replication.

\end{document}
