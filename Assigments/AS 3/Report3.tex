% Options for packages loaded elsewhere
\PassOptionsToPackage{unicode}{hyperref}
\PassOptionsToPackage{hyphens}{url}
%
\documentclass[
  12pt,
]{article}
\usepackage{amsmath,amssymb}
\usepackage{lmodern}
\usepackage{setspace}
\usepackage{iftex}
\ifPDFTeX
  \usepackage[T1]{fontenc}
  \usepackage[utf8]{inputenc}
  \usepackage{textcomp} % provide euro and other symbols
\else % if luatex or xetex
  \usepackage{unicode-math}
  \defaultfontfeatures{Scale=MatchLowercase}
  \defaultfontfeatures[\rmfamily]{Ligatures=TeX,Scale=1}
\fi
% Use upquote if available, for straight quotes in verbatim environments
\IfFileExists{upquote.sty}{\usepackage{upquote}}{}
\IfFileExists{microtype.sty}{% use microtype if available
  \usepackage[]{microtype}
  \UseMicrotypeSet[protrusion]{basicmath} % disable protrusion for tt fonts
}{}
\makeatletter
\@ifundefined{KOMAClassName}{% if non-KOMA class
  \IfFileExists{parskip.sty}{%
    \usepackage{parskip}
  }{% else
    \setlength{\parindent}{0pt}
    \setlength{\parskip}{6pt plus 2pt minus 1pt}}
}{% if KOMA class
  \KOMAoptions{parskip=half}}
\makeatother
\usepackage{xcolor}
\usepackage[margin=1in]{geometry}
\usepackage{graphicx}
\makeatletter
\def\maxwidth{\ifdim\Gin@nat@width>\linewidth\linewidth\else\Gin@nat@width\fi}
\def\maxheight{\ifdim\Gin@nat@height>\textheight\textheight\else\Gin@nat@height\fi}
\makeatother
% Scale images if necessary, so that they will not overflow the page
% margins by default, and it is still possible to overwrite the defaults
% using explicit options in \includegraphics[width, height, ...]{}
\setkeys{Gin}{width=\maxwidth,height=\maxheight,keepaspectratio}
% Set default figure placement to htbp
\makeatletter
\def\fps@figure{htbp}
\makeatother
\setlength{\emergencystretch}{3em} % prevent overfull lines
\providecommand{\tightlist}{%
  \setlength{\itemsep}{0pt}\setlength{\parskip}{0pt}}
\setcounter{secnumdepth}{-\maxdimen} % remove section numbering
\usepackage{amssymb}
\usepackage{amsmath}
\usepackage{subfig}
\usepackage{booktabs}
\ifLuaTeX
  \usepackage{selnolig}  % disable illegal ligatures
\fi
\IfFileExists{bookmark.sty}{\usepackage{bookmark}}{\usepackage{hyperref}}
\IfFileExists{xurl.sty}{\usepackage{xurl}}{} % add URL line breaks if available
\urlstyle{same} % disable monospaced font for URLs
\hypersetup{
  pdftitle={Econ771 - Empirical Exercise 3},
  pdfauthor={Nixon Torres Candiales},
  hidelinks,
  pdfcreator={LaTeX via pandoc}}

\title{Econ771 - Empirical Exercise 3}
\author{Nixon Torres Candiales}
\date{\today}

\begin{document}
\maketitle

\setstretch{1.25}
\hypertarget{overview}{%
\section{Overview}\label{overview}}

In this assignment, we're going to work through some applied issues
related to regression discontinuity designs. We'll cover the basics of
strict and fuzzy RD, and we'll work through standard specification
tests. We'll also introduce some more technical aspects of bin and
bandwidth selection.

Please ``submit'' your answers as a GitHub repository link on Canvas. In
this repo, please include a final document with your main answers and
analyses in a PDF. Be sure to include in your repository all of your
supporting code files. Practice writing good code and showing me only
what I would need to recreate your results.

\hypertarget{resources-and-data}{%
\section{Resources and data}\label{resources-and-data}}

The data for this assignment comes from the AEJ: Policy website, where
Keith Ericson's complete dataset is available. The data are available
\href{https://www.aeaweb.org/articles?id=10.1257/pol.6.1.38}{here}. I
will also upload the replication files to our class OneDrive folder.

\hypertarget{questions}{%
\section{Questions}\label{questions}}

In your GitHub repository, please be sure to clearly address/answer the
following questions.

\begin{enumerate}
\def\labelenumi{\arabic{enumi}.}
\tightlist
\item
  Recreate the table of descriptive statistics (Table 1) from
  @ericson2014.
\end{enumerate}

\begin{table}

\caption{Descriptive Statistics of Medicare Part D Plans}
\centering
\begin{tabular}[t]{lccccc}
\toprule
\multicolumn{1}{c}{ } & \multicolumn{5}{c}{Cohort (Year of plan introduction)} \\
\cmidrule(l{3pt}r{3pt}){2-6}
  & 2006 & 2007 & 2008 & 2009 & 2010\\
\midrule
Mean monthly premium & \$37 & \$40 & \$36 & \$30 & \$33\\
 & (13) & (17) & (20) & (5) & (9)\\
Mean deductible & \$92 & \$114 & \$146 & \$253 & \$118\\
 & (116) & (128) & (125) & (102) & (139)\\
Fraction enhanced benefit & 0.43 & 0.43 & 0.58 & 0.03 & 0.69\\
\addlinespace[0.3em]
\multicolumn{6}{l}{Fraction of plans offered by firms already offering a plan ... }\\
\hspace{1em}... in the United States & 0 & 0.76 & 0.98 & 1 & 0.97\\
\hspace{1em}... in the same state & 0 & 0.53 & 0.91 & 0.68 & 0.86\\
Number of unique firms & 51 & 38 & 16 & 5 & 6\\
Number of plans & 1429 & 658 & 202 & 68 & 107\\
\bottomrule
\end{tabular}
\end{table}
\pagebreak 

\begin{enumerate}
\def\labelenumi{\arabic{enumi}.}
\setcounter{enumi}{1}
\item
  Recreate Figure 3 from @ericson2014.

  \includegraphics{Output/fig/Q2.png}
\end{enumerate}

\newpage

\begin{enumerate}
\def\labelenumi{\arabic{enumi}.}
\setcounter{enumi}{2}
\item
  @calonico2015 discuss the appropriate partition size for binned
  scatterplots such as that in Figure 3 of Ericson (2014). More
  formally, denote by
  \(\mathcal{P}_{-,n} = \{ P_{-,j} : j=1, 2, ... J_{-, n} \}\) and
  \(\mathcal{P}_{+,n} = \{ P_{+,j} : j=1, 2, ... J_{+, n} \}\) the
  partitions of the support of the running variable \(x_{i}\) on the
  left and right (respectively) of the cutoff, \(\bar{x}\). \(P_{-, j}\)
  and \(P_{+, n}\) denote the actual supports for each \(j\) partition
  of size \(J_{-,n}\) and \(J_{+,n}\), such that
  \([x_{l}, \bar{x}) = \bigcup_{j=1}^{J_{-,n}} P_{-, j}\) and
  \((\bar{x}, x_{u}] = \bigcup_{j=1}^{J_{+,n}} P_{+, j}\). Individual
  bins are denoted by \(p_{-,j}\) and \(p_{+,j}\). With this notation in
  hand, we can write the partitions \(J_{-,n}\) and \(J_{+,n}\) with
  equally-spaced bins as
  \[p_{-,j}=x_{l} + j \times \frac{\bar{x} - x_{l}}{J_{-,n}},\] and
  \[p_{+,j} = \bar{x} + j \times \frac{x_{u} - \bar{x}}{J_{+,n}}.\]
  Recreate Figure 3 from Ericson (2014) using \(J_{-,n}=J_{+,n}=10\) and
  \(J_{-,n}=J_{+,n}=30\). Discuss your results and compare them to your
  figure in Part 2.

  \includegraphics{Output/fig/Q3A.png}

  \includegraphics{Output/fig/Q3B.png}
\item
  With the notation above, @calonico2015 derive the optimal number of
  partitions for an evenly-spaced (ES) RD plot. They show that
  \[J_{ES,-,n} = \left\lceil \frac{V_{-}}{\mathcal{V}_{ES,-}} \frac{n}{\text{log}(n)^{2}} \right\rceil\]
  and
  \[J_{ES,+,n} = \left\lceil \frac{V_{+}}{\mathcal{V}_{ES,+}} \frac{n}{\text{log}(n)^{2}} \right\rceil,\]
  where \(V_{-}\) and \(V_{+}\) denote the sample variance of the
  subsamples to the left and right of the cutoff and
  \(\mathcal{V}_{ES,.}\) is an integrated variance term derived in the
  paper. Use the \texttt{rdrobust} package in \texttt{R} (or
  \texttt{Stata} or \texttt{Python}) to find the optimal number of bins
  with an evenly-spaced binning strategy. Report this bin count and
  recreate your binned scatterplots from parts 2 and 3 based on the
  optimal bin number.
\end{enumerate}

\includegraphics{Output/fig/Q4A.png}
\includegraphics{Output/fig/Q4B.png} \pagebreak

\begin{enumerate}
\def\labelenumi{\arabic{enumi}.}
\setcounter{enumi}{4}
\item
\begin{verbatim}

Bandwidth selection for manipulation testing.

Number of obs =     4276
Model =             unrestricted
Kernel =            triangular
VCE method =        jackknife

Cutoff c = 0        Left of c           Right of c          
Number of obs       1944                2332                
Min Running var.    -31.24              0.01                
Max Running var.    0                   82.53               
Order est. (p)      2                   2                   

Target              Bandwidth           Variance            Bias^2              
left density        5.0449              0.3916              0                   
right density       3.832               0.3455              0                   
diff. densities     3.7562              0.7371              0                   
sum densities       6.0582              0.7371              0                   
\end{verbatim}

  \includegraphics{Output/fig/Q5.png} \newpage
\item
  Recreate Table 3 of @ericson2014 using the same bandwidth of
  \$4.00.\newline
\end{enumerate}

We recreate the table as follows, note the standard errors differs at
the third decimal. This discrepancy might be due to a different default
parameter in R when calculating the standard errors.

\begin{center}
\begin{table}
\caption{Effect of LIS Benchmark Status in 2006 on Plan Enrollment}
\centering
\begin{tabular}[t]{lccccc}
\toprule
 $\ln s_t$ & 2006 & 2007 & 2008 & 2009 & 2010\\
\midrule
\multicolumn{6}{l}{\textit{Panel A. Local linear, bandwidth \$4}}\\
Below benchmark, 2006 & 2.224*** & 1.332*** & 0.902** & 0.803* & 0.677\\
 & (0.283) & (0.267) & (0.248) & (0.362) & (0.481)\\
\addlinespace[0.3em]
\multicolumn{6}{l}{Premium—subsidy, 2006}\\
\hspace{1em}Below benchmark & -0.014 & -0.077 & -0.073 & -0.170 & -0.215*\\
\hspace{1em} & (0.032) & (0.088) & (0.116) & (0.105) & (0.088)\\
\hspace{1em}Above benchmark & -0.142+ & -0.033 & 0.049 & 0.074 & 0.049\\
\hspace{1em} & (0.078) & (0.110) & (0.163) & (0.170) & (0.202)\\
Num.Obs. & 306 & 299 & 298 & 246 & 212\\
R2 & 0.576 & 0.325 & 0.131 & 0.141 & 0.124\\
\\
\multicolumn{6}{l}{\textit{Panel B. Polynomial with controls, bandwidth \$4}}\\
Below benchmark, 2006 & 2.464*** & 1.364*** & 0.872** & 0.351 & -0.277\\
 & (0.219) & (0.317) & (0.243) & (0.321) & (0.298)\\
Premium--subsidy, 2006 & Quadratic & Quadratic & Quadratic & Quadratic & Quadratic\\
Num.Obs. & 306 & 299 & 298 & 246 & 212\\
R2 & 0.794 & 0.576 & 0.472 & 0.535 & 0.685\\

\bottomrule
\multicolumn{6}{l}{\rule{0pt}{1em}+ p $<$ 0.1, * p $<$ 0.05, ** p $<$ 0.01, *** p $<$ 0.001}\\
\end{tabular}
\end{table}
\end{center}

\newpage

\begin{enumerate}
\def\labelenumi{\arabic{enumi}.}
\setcounter{enumi}{6}
\item
  @calonico2020 show that pre-existing optimal bandwidth calculations
  (such as those used in @ericson2014) are invalid for appropriate
  inference. They propose an alternative method to derive minimal
  coverage error (CE)-optimal bandwidths. Re-estimate your RD results
  using the CE-optimal bandwidth (\texttt{rdrobust} will do this for
  you) and compare the bandwidth and RD estimates to that in Table 3 of
  @ericson2014.
\item
  Now let's extend the analysis in Section V of @ericson2014 using IV.
  Use the presence of Part D low-income subsidy as an IV for market
  share to examine the effect of market share in 2006 on future premium
  changes.
\item
  Discuss your findings and compare results from different binwidths and
  bandwidths. Compare your results in part 8 to the invest-then-harvest
  estimates from Table 4 in @ericson2014.
\item
  Reflect on this assignment. What did you find most challenging? What
  did you find most surprising?
\end{enumerate}

\hypertarget{references}{%
\section{References}\label{references}}

\end{document}
